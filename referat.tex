%\documentclass[a4pape, 14pt]{extarticle} % расширенный класс статьи с возможностью указать 14 шрифт
%\documentclass[a4pape, 11pt]{article}
\documentclass[oneside, final, 14pt]{extarticle}
\usepackage{graphicx}  % поддержка .eps-графики
\usepackage[utf8]{inputenc} % кодировка в которой набран текст 
\usepackage[T2C]{fontenc} % поддержка кирилицы % ещё можно T1 T2B T2A T2
\usepackage[english,russian]{babel} % переносы и типографские правила для русского
\usepackage{indentfirst}  % красная строка
\usepackage{listings}  % оформление листингов программ
\usepackage{amssymb,amsfonts,amsmath,mathtext} % ядро для научной статьи
\usepackage{physics}
\usepackage{mathtools}
\usepackage{braket}
\usepackage{float} % для рисунков
\usepackage{array} % для m{2cm} в таблицах
\usepackage{blindtext} % случайный текст для заглушки

\usepackage{titlesec}
%\linespread{1.3}
%\usepackage{setspace}\onehalfspacing
%\usepackage{vmargin} отступы из книги столярова
%\setpapersize{A4}
%\setmarginsrb{2.5cm}{2cm}{1.5cm}{2cm}{0pt}{0mm}{0pt}{13mm}


\usepackage{pdfpages} % для вставки пдф файлов

\usepackage{pgfplots} % построение графиков
% We will externalize the figures
%\usepgfplotslibrary{external}
%\tikzexternalize
%\pgfplotsset{width=15cm,compat=1.5}

%\sloppy
%\fussy

\usepackage[left=2cm, right=2cm, top=1.5cm, bottom=1.5cm]{geometry} % установка полей

\newcommand{\hup}{Принцип неопределенности Гейзенберга}
\newcommand{\qe}{Квантовая запутанность}

%\usepackage{citehack}  % https://www.opennet.ru/docs/RUS/cyr_howto/ch08s02.html
%\usepackage[bibencoding=auto,backend=biber,babel=other]{biblatex}
%\usepackage{csquotes} 


% https://ru.overleaf.com/learn/latex/Pgfplots_package  - для построения графиков
% https://ru.overleaf.com/learn/latex/Tables - построение таблиц
% https://ru.overleaf.com/learn/latex/Commands - обьявление коман
\graphicspath{{./images/}} %путь к рисункам
\frenchspacing % длина пробелов после пунктуации
\pagestyle{plain}
\selectlanguage{russian}
\bibliographystyle{gost780} % gost780 - сортировка по порядку следования, gost780s - по алфавиту кажется. gost780u не завелся

\begin{document}

\begin{titlepage}
\centerline{\includegraphics[width=2cm]{dstu-logo}}
\vfill
%\Large
\centerline{МИНЕСТЕРСТВО НАУКИ И ВЫСШЕГО ОБРАЗОВАНИЯ}
\centerline{РОССИЙСКОЙ ФЕДЕРАЦИИ}
\vfill
\centerline{\bf ФЕДЕРАЛЬНОЕ ГОСУДАРСТВЕННОЕ БЮДЖЕТНОЕ}
\centerline{\bf ОБРАЗОВАТЕЛЬНОЕ УЧРЕЖДЕНИЕ ВЫСШЕГО ОБРАЗОВАНИЯ}
\centerline{\bf «ДОНСКОЙ ГОСУДАРСТВЕННЫЙ ТЕХНИЧЕСКИЙ УНИВЕРСИТЕТ»}
\centerline{\bf (ДГТУ)}
\normalsize
\vfill\vfill
\centerline{Факультет «Отдел аспирантуры и докторантуры»}
\centerline{Кафедра «Кибербезопасность информационных систем»}
\vfill
%\rightline{Место для подписи завкафедрой}
\vfill
%\Large{\centerline{\bf ПОРТФОЛИО АСПИРАНТА}}
\centerline{Реферат по дисциплине «Проектный менеджмент в науке и технологиях»}
\centerline{на тему: Развитие протоколов квантового распределения ключей}
\vfill
\vfill
\vfill
\vfill
\vfill
\vfill
\vfill
\vfill
\rightline{Выполнил:}
\rightline{аспирант 1го курса}
\rightline{по направлению подготовки}
\rightline{09.06.01 «Информатика и вычислительная техника»}% номер зачетной книжки: 2142748
\rightline{Галов К. А.}
\rightline{Проверил: д.э.н., доцент Змияк С. С.}
\vfill
\vfill

\centerline{г. Ростов-на-Дону}
\centerline{2022 г.}

\end{titlepage}
\setcounter{page}{2}

\section*{Введение}
Протокол квантового распределения ключей - это криптографический алгоритм, основанный свойствах квантовых объектов:

\begin{itemize}
\item принципе неопределенности Гейзенберга;
\item квантовой запутанности.
\end{itemize}

Криптография играет важную роль в нашей безопастности, защищая от злоумышленника конфиденциальную информацию. Однако с развитием квантовых технологий и разработкой квантовых компьютеров требуются новые способы защиты информации.

Стойкость современной классической криптографии основана на предположении о том, что вычислительные мощности злоумышлинника ограниченны. Это основание имеет ряд недостатков:

\begin{itemize}
\item Вычислительная техника быстро развивается. Неизвестно какими вычислительными мощностями будут обладать компьютеры в ближайшем будущем.
\item Неизвестно отношение классов задач P и NP. Если классы задач P и NP равны, то существует эффективный алгоритм для взлома классических шифров.
\item Разработка квантовых компьютеров. Квантовые компьютеры в силу своих свойств способны эффективно решать задачи с экспоненциальной сложностью.
\end{itemize}

В свою очередь, стойкость протоколов квантового распределения ключей основывается на нерушимых законах физики. Злоумышленник не может остаться незамеченным при прослушивания.

Однако, квантовая криптография также имеет ряд недостатков. В силу особенностей технической реализации, злоумышленник может провести специальну атаку с разделением числа фотонов и получить доступ к передаваемой информации. Поэтому развитие протоколов квантового распределение ключей является важной и актуальной задачей.


\section{Ch H. B., Brassard G. Quantum cryptography: public key distribution and coin tossing int //Conf. on Computers, Systems and Signal Processing (Bangalore, India. – 1984. – Т. 175.}
На конференции в 1984 году в работе \cite{ch1984quantum} был представлен первый протокол квантового распределения ключей.

В своей работе авторы использовали два канала связи: классический публичный и квантовый, по которому передовались фотоны света, поляризованные под углами 0, 45, 90 и 135 градусов. При попытке прослушать квантовый канал, злоумышленник неизбежно вносил шум в канал связи, так как невозможно достоверно отличить фотоны в неортогонольных состояниях.

Таким образом, авторы работы показали как с помощью свойств квантовых объектов:
\begin{itemize}
	\item определить наличие в канале злоумышленника;
	\item сгенерировать случайное число между собеседниками, которое в дальнейшем может быть использованно для создания ключа шифрования или в качестве одноразового шифр-блокнота.
\end{itemize}

\section{Ekert A. K. Quantum cryptography based on Bell’s theorem //Physical review letters. – 1991. – Т. 67. – №. 6. – С. 661.}

\section{Huttner B. et al. Quantum cryptography with coherent states //Physical Review A. – 1995. – Т. 51. – №. 3. – С. 1863.}
\blindtext


\section{Gisin N. et al. Towards practical and fast quantum cryptography //arXiv preprint quant-ph/0411022. – 2004.}
\blindtext

\section{Khan M. M., Murphy M., Beige A. High error-rate quantum key distribution for long-distance communication //New Journal of Physics. – 2009. – Т. 11. – №. 6. – С. 063043. }
\blindtext
\section{Lo H. K., Ma X., Chen K. Decoy state quantum key distribution //Physical review letters. – 2005. – Т. 94. – №. 23. – С. 230504.}
\blindtext
\section{Singh H., Gupta D. L., Singh A. K. Quantum key distribution protocols: a review //Journal of Computer Engineering. – 2014. – Т. 16. – №. 2. – С. 1-9.}
\blindtext
\section{Serna E. H. Quantum key distribution protocol with private-public key //arXiv preprint arXiv:0908.2146. – 2009.}
\blindtext
\section{Scarani V. et al. Quantum cryptography protocols robust against photon number splitting attacks for weak laser pulse implementations //Physical review letters. – 2004. – Т. 92. – №. 5. – С. 057901.}
\blindtext
\section{Song D., Chen D. Quantum Key Distribution Based on Random Grouping Bell State Measurement //IEEE Communications Letters. – 2020. – Т. 24. – №. 7. – С. 1496-1499.}
\blindtext
\section{Душкин Р. В. Обзор текущего состояния квантовых технологий //Компьютерные исследования и моделирование. – 2018. – Т. 10. – №. 2. – С. 165-179.}

\section*{Заключение}
\blindtext

\bibliography{refs}
\end{document}