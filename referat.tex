%\documentclass[a4pape, 14pt]{extarticle} % расширенный класс статьи с возможностью указать 14 шрифт
%\documentclass[a4pape, 11pt]{article}
\documentclass[oneside, final, 14pt]{extarticle}
\usepackage{graphicx}  % поддержка .eps-графики
\usepackage[utf8]{inputenc} % кодировка в которой набран текст 
\usepackage[T2C]{fontenc} % поддержка кирилицы % ещё можно T1 T2B T2A T2
\usepackage[english,russian]{babel} % переносы и типографские правила для русского
\usepackage{indentfirst}  % красная строка
\usepackage{listings}  % оформление листингов программ
\usepackage{amssymb,amsfonts,amsmath,mathtext} % ядро для научной статьи
\usepackage{physics}
\usepackage{mathtools}
\usepackage{braket}
\usepackage{float} % для рисунков
\usepackage{array} % для m{2cm} в таблицах
\usepackage{blindtext} % случайный текст для заглушки

\usepackage{titlesec}
%\linespread{1.3}
%\usepackage{setspace}\onehalfspacing
%\usepackage{vmargin} отступы из книги столярова
%\setpapersize{A4}
%\setmarginsrb{2.5cm}{2cm}{1.5cm}{2cm}{0pt}{0mm}{0pt}{13mm}


\usepackage{pdfpages} % для вставки пдф файлов

\usepackage{pgfplots} % построение графиков
% We will externalize the figures
%\usepgfplotslibrary{external}
%\tikzexternalize
%\pgfplotsset{width=15cm,compat=1.5}

%\sloppy
%\fussy

\usepackage[left=2cm, right=2cm, top=1.5cm, bottom=1.5cm]{geometry} % установка полей

\newcommand{\hup}{Принцип неопределенности Гейзенберга}
\newcommand{\qe}{Квантовая запутанность}

%\usepackage{citehack}  % https://www.opennet.ru/docs/RUS/cyr_howto/ch08s02.html
%\usepackage[bibencoding=auto,backend=biber,babel=other]{biblatex}
%\usepackage{csquotes} 


% https://ru.overleaf.com/learn/latex/Pgfplots_package  - для построения графиков
% https://ru.overleaf.com/learn/latex/Tables - построение таблиц
% https://ru.overleaf.com/learn/latex/Commands - обьявление коман
\graphicspath{{./images/}} %путь к рисункам
\frenchspacing % длина пробелов после пунктуации
\pagestyle{plain}
\selectlanguage{russian}
\bibliographystyle{gost780} % gost780 - сортировка по порядку следования, gost780s - по алфавиту кажется. gost780u не завелся

\begin{document}

\begin{titlepage}
\centerline{\includegraphics[width=2cm]{dstu-logo}}
\vfill
%\Large
\centerline{МИНЕСТЕРСТВО НАУКИ И ВЫСШЕГО ОБРАЗОВАНИЯ}
\centerline{РОССИЙСКОЙ ФЕДЕРАЦИИ}
\vfill
\centerline{\bf ФЕДЕРАЛЬНОЕ ГОСУДАРСТВЕННОЕ БЮДЖЕТНОЕ}
\centerline{\bf ОБРАЗОВАТЕЛЬНОЕ УЧРЕЖДЕНИЕ ВЫСШЕГО ОБРАЗОВАНИЯ}
\centerline{\bf «ДОНСКОЙ ГОСУДАРСТВЕННЫЙ ТЕХНИЧЕСКИЙ УНИВЕРСИТЕТ»}
\centerline{\bf (ДГТУ)}
\normalsize
\vfill\vfill
\centerline{Факультет «Отдел аспирантуры и докторантуры»}
\centerline{Кафедра «Кибербезопасность информационных систем»}
\vfill
%\rightline{Место для подписи завкафедрой}
\vfill
%\Large{\centerline{\bf ПОРТФОЛИО АСПИРАНТА}}
\centerline{Реферат по дисциплине «Проектный менеджмент в науке и технологиях»}
\centerline{на тему: Развитие протоколов квантового распределения ключей}
\vfill
\vfill
\vfill
\vfill
\vfill
\vfill
\vfill
\vfill
\rightline{Выполнил:}
\rightline{аспирант 1го курса}
\rightline{по направлению подготовки}
\rightline{09.06.01 «Информатика и вычислительная техника»}% номер зачетной книжки: 2142748
\rightline{Галов К. А.}
\rightline{Проверил: д.э.н., доцент Змияк С. С.}
\vfill
\vfill

\centerline{г. Ростов-на-Дону}
\centerline{2022 г.}

\end{titlepage}
\setcounter{page}{2}

\section*{Введение}
Протокол квантового распределения ключей - это криптографический алгоритм, основанный на следующих свойствах квантовых объектов:

\begin{itemize}
\item принципе неопределенности Гейзенберга;
\item квантовой запутанности.
\end{itemize}

Криптография играет важную роль в нашей безопасности, защищая от злоумышленника конфиденциальную информацию. Однако с развитием квантовых технологий и разработкой квантовых компьютеров требуются новые способы защиты информации.

Стойкость современной классической криптографии основана на предположении о том, что вычислительные мощности злоумышленника ограниченны. Это основание имеет ряд недостатков:

\begin{itemize}
\item Вычислительная техника быстро развивается. Неизвестно какими вычислительными мощностями будут обладать компьютеры в ближайшем будущем.
\item Неизвестно отношение классов задач P и NP. Если классы задач P и NP равны, то существует эффективный алгоритм для взлома классических шифров.
\item Разработка квантовых компьютеров. Квантовые компьютеры в силу своих свойств способны эффективно решать задачи с экспоненциальной сложностью.
\end{itemize}

В свою очередь, стойкость протоколов квантового распределения ключей основывается на нерушимых законах физики. Злоумышленник не может остаться незамеченным при прослушивании.

Однако, квантовая криптография также имеет ряд недостатков. В силу особенностей технической реализации, злоумышленник может провести специальную атаку с разделением числа фотонов и получить доступ к передаваемой информации. Поэтому развитие протоколов квантового распределение ключей является важной и актуальной задачей.


\section{Ch H. B., Brassard G. Quantum cryptography: public key distribution and coin tossing int //Conf. on Computers, Systems and Signal Processing (Bangalore, India. – 1984. – Т. 175.}
На конференции в 1984 году в работе \cite{ch1984quantum} был представлен первый протокол квантового распределения ключей.

В своей работе авторы использовали два канала связи: классический публичный и квантовый, по которому передавались фотоны света, поляризованные под углами 0, 45, 90 и 135 градусов. При попытке прослушать квантовый канал, злоумышленник неизбежно вносил шум в канал связи, так как невозможно достоверно отличить фотоны в неортогонольных состояниях.

Таким образом, авторы работы показали как с помощью свойств квантовых объектов:
\begin{itemize}
	\item определить наличие в канале злоумышленника;
	\item сгенерировать случайное число между собеседниками, которое в дальнейшем может быть использовано для создания ключа шифрования или в качестве одноразового шифр-блокнота.
\end{itemize}

\section{Ekert A. K. Quantum cryptography based on Bell’s theorem //Physical review letters. – 1991. – Т. 67. – №. 6. – С. 661.}
Протокол Е91, опубликованный в работе \cite{ekert1991quantum}, представляет вторую группу протоколов квантового распределения ключей.

Данный протокол основывается на явлении квантовой запутанности. Суть явления заключается в том, что при измерении состояния запутанных квантовых объектов, результаты измерений оказываются прямо противоположны. Измерение разрушает состояние квантовых объектов и не существует способа узнать о состоянии объекта до его измерения.

Авторы работы предложили использовать специальный лазер в качестве источника запутанных фотонов. При попытке прочитать запутанный фотон, злоумышленник разрушает связь между фотонами и получатель, при помощи неравенства Белла, может определить наличие в канале связи злоумышленника.


\section{Huttner B. et al. Quantum cryptography with coherent states //Physical Review A. – 1995. – Т. 51. – №. 3. – С. 1863.}
Протоколы BB84 и E91 обладают общей уязвимостью - атака с разделением числа фотонов. Современные лазеры, которые используются в качестве источника фотонов, не идеальны. За один раз они могут испустить более одного фотона. Это позволяет злоумышленнику перехватить часть фотонов и получить доступ к конфиденциальной информации без внесения дополнительного шума в канал.

В работе \cite{huttner1995quantum} авторами был представлен протокол 4+2, в котором они попытались решить проблему атаки с разделением числа фотонов. Авторы взяли за основу протокол BB84 и модифицировали его, взяв идеи протокола Е92. В результате комбинирования протоколов, фотоны требовалось поляризовывать под особыми комбинациями углов, что не давало злоумышленнику получить доступ к конфиденциальной информации при перехвате фотона.

Однако в дальнейших работах была показана модифицированная атака с разделением числа фотонов, которая разрушала защиту данного протокола.


\section{Scarani V. et al. Quantum cryptography protocols robust against photon number splitting attacks for weak laser pulse implementations //Physical review letters. – 2004. – Т. 92. – №. 5. – С. 057901.}

В данной работе \cite{scarani2004quantum} авторы начинают с критики протокола 4+2. Они показывают, что применяя специальное преобразование фильтрации, можно свести атаку на протокол 4+2 к атаке на протокол BB84.

Далее авторы статьи модифицируют протокол 4+2. Для защиты от преобразования фильтрации они используют комбинацию базисов поляризации фотонов, которая не связанна унитарным преобразованием. В таком случае проделать фильтрацию становится невозможно.

В результате получился протокол, обладающий повышенной устойчивостью к атаке с разделением числа фотонов. Для получения доступа к конфиденциальной информации злоумышленнику необходимо перехватить все двух- и трёхфотонные посылки. Таким образом, протокол SARG04 не дает полной безопасности, но является более безопасным, чем протокол BB84.



\section{Gisin N. et al. Towards practical and fast quantum cryptography //arXiv preprint quant-ph/0411022. – 2004.}
Авторы в работе \cite{gisin2004towards} представили новый протокол для практической квантовой криптографии, адаптированный для реализации со слабыми когерентными импульсами. Данный протокол имеет существенные отличия от предыдущих.

Ключ получается при помощи измерения времени прихода данных. При помощи интерферомета на дополнительной линии контроля, появляется возможность отслеживать присутствие шпиона, который своим вмешательством нарушит когерентность фотонов. Против атак с нулевой ошибкой (аналог атак с расщеплением фотонного числа) этот протокол работает так же хорошо, как и стандартные протоколы.

В статье представлены две атаки, которые вносят ошибки на линии: перехват-посылка и когерентная атака на два последующих импульса.

Также авторы предоставили несколько возможных практических реализаций данного протокола.


\section{Lo H. K., Ma X., Chen K. Decoy state quantum key distribution //Physical review letters. – 2005. – Т. 94. – №. 23. – С. 230504.}
Некоторые ученые ошибочно считают, что в данной \cite{lo2005decoy} работе был представлен новый протокол, однако на самом деле авторы показали новый метод квантовых ловушек. Авторы статьи использовали метод вместе с протоколом BB84, но они также утверждают, что его можно использовать с любым протоколом квантового распределения ключей.

Суть метода заключается в том, что вместе с фотонами, содержащими конфиденциальную информацию, по каналу связи отправляются фотоны-ловушки. Они не несут в себе содержательной информации. Когда злоумышленник считывает фотон-ловушку, он увеличивает количество шума в канале связи, тем самым выдавая себя, и не получает никакой конфиденциальной информации.

В результате авторы смогли увеличить дистанцию безопасной передачи информации по квантовому каналу связи с 30км до 150км.

\section{Singh H., Gupta D. L., Singh A. K. Quantum key distribution protocols: a review //Journal of Computer Engineering. – 2014. – Т. 16. – №. 2. – С. 1-9.}
Данная статья \cite{singh2014quantum} является обзорной. Авторы описывают основные физические явления, лежащие в основе протоколов, и делят их на две категории:
\begin{itemize}
	\item основанные на принципе неопределенности Гейзенберга;
	\item основанные на явлении квантовой запутанности.
\end{itemize}

Затем авторы рассматривают протоколы, начиная с самых первых (BB84, E91) и продолжая до новейших (S13). Эта работа показывает интерес научной среды в изучении протоколов квантового распределения ключей.

Результатом работы является сравнительная таблица всех протоколов, в которой авторы выделяют основные особенности и отличия.

\section{Душкин Р. В. Обзор текущего состояния квантовых технологий //Компьютерные исследования и моделирование. – 2018. – Т. 10. – №. 2. – С. 165-179.}
В этой работе \cite{dushkin2018obzor} российского ученого рассматриваются многие квантовые технологии:
\begin{itemize}
	\item квантовая передача информации;
	\item квантовая сенсорика;
	\item квантовый компьютер;
	\item квантовое программное обеспечение.
\end{itemize}
Автор описывает последние достижения в данных областях и делает выводы о дальнейшем развитии квантовых технологий.


%\section{Khan M. M., Murphy M., Beige A. High error-rate quantum key distribution for long-distance communication //New Journal of Physics. – 2009. – Т. 11. – №. 6. – С. 063043. }
%\blindtext


%\section{Serna E. H. Quantum key distribution protocol with private-public key //arXiv preprint arXiv:0908.2146. – 2009.}
%\blindtext

%\section{Song D., Chen D. Quantum Key Distribution Based on Random Grouping Bell State Measurement //IEEE Communications Letters. – 2020. – Т. 24. – №. 7. – С. 1496-1499.}
%\blindtext

\clearpage

\section*{Заключение}
\thispagestyle{empty}
Таким образом, за последние тридцать лет наблюдается повышенный интерес к протоколам квантового распределения ключей и к квантовым технологиям в целом. Несмотря на популярность данной сферы науки, она остается плохо изученной. В данный момент не существует полностью надежного протокола квантового распределения ключей.

Актуальность данной темы обусловлена широтой проблемы защиты конфиденциальной информации. Ее изучение позволит создать невзламываемую криптографию, которая будет основана на фундаментальных законах физики, а не на неподтвержденных математических гипотезах.

%\clearpage

\bibliography{refs}

\end{document}